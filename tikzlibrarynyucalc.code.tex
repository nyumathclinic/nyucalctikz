%%
%% This is file `tikzlibrarynyucalc.code.tex',
%% generated with the docstrip utility.
%%
%% The original source files were:
%%
%% nyucalctikz.dtx  (with options: `library')
%% -----------------------------------------------------------------------------
%% tikzlibrarynyucalc.code.tex -- a TikZ/PGF library for NYU calculus materials
%% E-mail: leingang@nyu.edu
%% All rights reserved.
%% -----------------------------------------------------------------------------
%% 
\NeedsTeXFormat{LaTeX2e}
\ProvidesFile{tikzlibrarynyucalc.code.tex}[2017/03/03 v0.0 TikZ and PGFPlots code for NYU calculus materials]
\usepackage{tikz}
\input{svgnam.def}
\usetikzlibrary{%
    positioning,
    shapes,
}
%%
\tikzstyle{point}=[
    fill,fill opacity=1,
    circle,
    minimum size=3pt,inner sep=0pt,outer sep=0pt]
\tikzstyle{open point}=[draw,fill=white,circle,minimum size=3pt,inner sep=0pt,outer sep=0pt]
\pgfdeclarelayer{background}
\pgfdeclarelayer{foreground}
\pgfsetlayers{background,main,foreground}

\tikzstyle{measurement}=[arrows={|<->|},Black]
\tikzstyle{angle measurement}=[measurement,arrows={->}]
\tikzstyle{behind}=[black!20]
\tikzstyle{imaginary}=[dashed]
\tikzstyle{set}=[draw=primary,thick,ellipse,shade,top color=primary!50,bottom color=white,inner sep=4mm]
\tikzstyle{chain}=[%
    decoration={%
        shape backgrounds,
        shape=ellipse,
        shape width=5pt,
        shape height=2.5pt,
        shape sep=4pt},
    decorate
]
\colorlet{primary}{blue!40!black}       % HTML {000066}
\colorlet{secondary}{blue!20!white}     % HTML {CCCCFF}
\colorlet{tertiary}{Green!40!Black}     % HTML {006600}
\colorlet{quartenary}{Green!20!white}   % HTML {CCFFCC}
\definecolor{quintenary}{HTML}{1C1F24}
\tikzstyle{curve}=[primary,thick]
\tikzstyle{curve label}=[black,opacity=1]
\tikzstyle{curve direction}=[curve,fill opacity=1]
\tikzstyle{region}=[curve,fill=quartenary]
\tikzstyle{surface}=[fill=quartenary,opacity=0.5]
\tikzstyle{function}=[curve]
\RequirePackage{pgfplots}
\pgfplotsset{compat=1.10}
\usepgfplotslibrary{groupplots}
\usepgfplotslibrary{fillbetween}
\pgfplotsset{
    every linear axis/.append style={%
        axis x line=middle, axis y line=middle,%
        cycle multi list={%
            solid,dashed,dotted\nextlist
            {function,primary},
            {function,secondary},
            {function,tertiary},
            {function,quartenary}
        },
    },
    every axis plot/.append style={every label/.append style={black}}
}
\tikzstyle{every mark}=[mark=*,mark size=1pt]
\pgfplotsset{
    number line/.style={%
        axis x line=bottom,
        ymin=0,ymax=0.1,
        axis equal image,
        hide y axis,
        every axis x label/.append style={%
            anchor=west,
            align=left,
        },
        execute at end axis={
            \addplot[draw=none,forget plot] {0};
        }
    }
}
\pgfplotsset{%
    number line y/.style={%
        axis y line=left,
        axis equal image,
        xmin=0,xmax=0.1,
        hide x axis,
        every axis y label/.append style={
            anchor=south east,
            align=left
        },
    }
}
\pgfplotsset{%
    interval labels/.style={
        after end axis/.prefix code={
            \foreach \x/\val/\desc in {#1} {
                \ifx\val\desc
                    \edef\temp{\noexpand\draw ({axis cs:\x,0}|-{xticklabel* cs:0})
                        node[number line value] {\unexpanded\expandafter{\val}};}%
                \else
                    \edef\temp{\noexpand\draw ({axis cs:\x,0}|-{xticklabel* cs:0})
                    node[number line value] {\unexpanded\expandafter{\val}}
                    node[number line description] {\unexpanded\expandafter{\desc}};}
                \fi
                \temp
            }
        }
    },
    point labels/.style={
        after end axis/.prefix code={
            \foreach \x/\val/\desc in {#1} {
                \ifx\val\desc
                    \edef\temp{\noexpand\draw ({axis cs:\x,0} |- {xticklabel* cs:0})
                        node[number line value] {\unexpanded\expandafter{\val}};}%
                \else
                    \edef\temp{%
                        \noexpand\draw ({axis cs:\x,0} |- {xticklabel* cs:0})
                            node[number line value] {\unexpanded\expandafter{\val}};
                        \noexpand\draw ({axis cs:\x,0} |- {xticklabel cs:0})
                        node[number line description] {\unexpanded\expandafter{\desc\strut}};
                        }%
                \fi
                \temp
            }
        }
    }
}
\usepgfplotslibrary{polar}
\pgfplotsset{slope field/.style={
    secondary,thin,
    /pgfplots/quiver={u=1,v={#1},scale arrows=0.1},
    /pgfplots/x filter/.expression={x-1*0.5*\pgfkeysvalueof{/pgfplots/quiver/scale arrows}},
    /pgfplots/y filter/.expression={y-(#1)*.5*\pgfkeysvalueof{/pgfplots/quiver/scale arrows}}
    },
    ode solution/.style={primary},
}
\tikzset{>=stealth}
\tikzstyle{vector}=[primary,very thick,arrows={->}]
\tikzstyle{vector label}=[pos=0.5,black]
%% 
%% Copyright (C) 2012-7 by Matthew Leingang <leingang@nyu.edu>
%% and the NYU Math Clinic.
%% 
%% This work consists of the file  nyucalctikz.dtx
%% and the derived files           tikzlibrarynyucalc.code.tex
%%
%% End of file `tikzlibrarynyucalc.code.tex'.
