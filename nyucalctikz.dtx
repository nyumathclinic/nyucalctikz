% \iffalse meta-comment
% http://www.texdev.net/2009/10/06/a-model-dtx-file/
% !TEX program  = pdfLaTeX
%<*internal>
\iffalse
%</internal>
%<*internal>
\fi
\def\nameofplainTeX{plain}
\ifx\fmtname\nameofplainTeX\else
  \expandafter\begingroup
\fi
%</internal>
%<*install>
\input docstrip.tex
\keepsilent
\askforoverwritefalse
\preamble
-----------------------------------------------------------------------------
tikzlibrarynyucalc.code.tex -- a TikZ/PGF library for NYU calculus materials
E-mail: leingang@nyu.edu
All rights reserved.
-----------------------------------------------------------------------------

\endpreamble
\postamble

Copyright (C) 2012-7 by Matthew Leingang <leingang@nyu.edu> 
and the NYU Math Clinic.

This work consists of the file  nyucalctikz.dtx
and the derived files           tikzlibrarynyucalc.code.tex
\endpostamble
%<install> \UseTDS
%<install> \usedir{tex/generic}
%<*!install>
\ifx\fmtname\nameofplainTeX
%</!install>
\generate{\file{tikzlibrarynyucalc.code.tex}{\from{\jobname.dtx}{library}}}
%<*!install>
\generate{\file{\jobname.ins}{\from{\jobname.dtx}{install}}}
\fi
%</!install>
%</install>
%<install>\end
%<*internal>
\ifx\fmtname\nameofplainTeX
  \expandafter\endbatchfile\else
  \expandafter\endgroup\fi 
%</internal>
%<*library>
% Keep version numbering separate from nyucalc.tex version numbering
\NeedsTeXFormat{LaTeX2e}
\ProvidesFile{tikzlibrarynyucalc.code.tex}[2017/03/02 v0.0-dev TikZ and PGFPlots code for NYU calculus materials]
%</library>
%<*driver>
\documentclass{amsdtx}
\usepackage{tikzdtx}
\usepackage[letterpaper]{geometry}
\usepackage[numbered]{hypdoc}
\usepackage{helvet}
\renewcommand{\familydefault}{\sfdefault}
\usepackage{tikz}
\usetikzlibrary{nyucalc}
\usetikzlibrary{fit}
\usepackage{pgfcodex}
\def\tikz/{\pkg{\protect{\normalfont Ti\textit{k}Z}}}
\def\pgf/{\pkg{\protect{\normalfont\textsc{pgf}}}}
\def\pgfplots/{\pkg{\protect{\normalfont\textsc{pgfplots}}}}
\EnableCrossrefs
\CodelineIndex
\RecordChanges
\begin{document}
  \DocInput{\jobname.dtx}
\end{document}
%</driver>
% \fi
%
% \GetFileInfo{tikzlibrarynyucalc.code.tex}
%
% \changes{v0.0}{2017/03/02}{Copied from nyucalctikz}
%
% \title{A TikZ library for NYU calculus materials}
% \author{Matthew Leingang\thanks{Email: leingang@nyu.edu}}
% \date{Released \filedate\thanks{Version: \fileversion}}
%
% \maketitle
% 
% \tableofcontents
%
% \StopEventually{^^A
%   \PrintChanges
%   \PrintIndex
% }
%
% \setcounter{section}{-1}
% \section{Introduction}
%
% This library grew out of a file used for \pkg{beamer} presentations for a 
% Calculus I class.  Eventually keys and styles were added for Calculus II and III.
% The file became the basis of a proprietary \tikz/ library for \emph{Calculus 
% for the Digital Student}.
% 
% After several years the need arose to share with the rest of the NYU Mathematics
% Clinical Faculty.  The \pkg{beamer} parts were excised out and the \tikz/ parts
% remain.
%
% This documentation file includes the implementation code and several examples
% to aid in making nice diagrams and graphs.
%
%
% We use \tikz/ and \pgfplots/, both built on the \pgf/ package.
% This code goes into the file \fn{tikzlibrarynyucalc.code.tex}.
% \setcounter{CodelineNo}{0}
%    \begin{macrocode}
%<*library>
%    \end{macrocode}
%
% \section{TikZ}
%
%    \begin{macrocode}
\usepackage{tikz}
%    \end{macrocode}
% 
% We want the colors defined with the \opt{svgnames} options of the \pkg{xcolor} package.
% Since \tikz/ already loads \pkg{xcolor}, we have to go one step down and input the color definition file.
%    \begin{macrocode}
\input{svgnam.def}
%    \end{macrocode}
%
% The following \tikz/ libraries are required:
% \begin{description}
%    \item[\pkg{positioning}]     for positioning nodes relative to each other
%    \item[\pkg{shapes}]          for special node shapes
% \end{description}
%
% The following \tikz/ libraries are required to typeset the manual, but not
% to use the library.
% \begin{description}
%    \item[\pkg{fit}]             for fitting nodes around nodes
% \end{description}
%
%
% Several others are useful, but not required by this library.
% As the manual grows, these may migrate into the section above.
% \begin{description}
%     \item[\pkg{hobby}]           for drawing nice curves through points
%     \item[\pkg{patterns}]        for filling with patterns (Macintosh '86 style)
%     \item[\pkg{decorations.text}] for decorating paths with text
%     \item[\pkg{decorations.shapes}] for decorating paths with shapes
%     \item[\pkg{spy}]             for creating spy-on/spy-in nodes
%     \item[\pkg{intersections}]   for finding intersections of paths
%     \item[\pkg{arrows}]          for various arrow shapes 
%     \item[\pkg{calc}]            for calculations on dimensions
%     \item[\pkg{matrix}]          for creating matrices of TikZ nodes 
% \end{description}
%    \begin{macrocode}
\usetikzlibrary{%
    positioning,
%    fit,
    shapes,
}
%    \end{macrocode}
%
%%
% \subsection{Graphs} 
%
% These two node styles are used still.
% They could be someday replaced with \pgfplots/ decorations (\S4.16.3 of the manual)
% or \cn{addplot} with markers and coordinates?
% \verb|mark=*| and \verb|mark=o| ought be useful here.
% 
%    \begin{macrocode}
\tikzstyle{point}=[
    fill,fill opacity=1,
    circle,
    minimum size=3pt,inner sep=0pt,outer sep=0pt]
\tikzstyle{open point}=[draw,fill=white,circle,minimum size=3pt,inner sep=0pt,outer sep=0pt]
%    \end{macrocode}
%
% \subsection{Layers} 
% With \pgf/ layers we can draw later objects under earlier objects.
%    \begin{macrocode}
\pgfdeclarelayer{background}
\pgfdeclarelayer{foreground}
\pgfsetlayers{background,main,foreground}
%    \end{macrocode}
% 
%
% \subsection{Miscellaneous styles}
% April 14, 2012: Nancy said the \tk{measurement} style is easier to see when it's black.
% \begin{center}
%   \begin{tikzpicture}    
%     \tikzstyle{board}=[draw=BurlyWood!50!black,fill=BurlyWood]
%     % left stud
%     \draw[board] (0,-1) rectangle (0.3,2);
%     \draw[xshift=-0.1cm,measurement] (0,-1) -- (0,2) node[midway,left] {$a$};
%     % right stud
%     \draw[board,xshift=2cm] (0,-1) rectangle (0.3,2.2);
%     \draw[xshift=2.4cm,measurement] (0,-1) -- (0,2.2) node[midway,right] {$x$};
%     % header
%     \draw[board,rotate around={5:(0,2)}] (0,2) rectangle (2.3,2.3);
%     \node[above,sloped,rotate around={5:(0,2)}] at (1,2.3) {header};
%     \draw[measurement,arrows={-}] ([rotate around={5:(0,2)}]2.3,2) -- (0,2) -- (2.3,2)
%       node[coordinate,
%         pin={[inner sep=0pt,pin edge={measurement,arrows=<-}]-135:$\theta$}
%       ] at ([rotate around={2.5:(0,2)}]1.75,2) {};
%   \end{tikzpicture}    
% \end{center}

%    \begin{macrocode}
\tikzstyle{measurement}=[arrows={|<->|},Black]
\tikzstyle{angle measurement}=[measurement,arrows={->}]
\tikzstyle{behind}=[black!20]
\tikzstyle{imaginary}=[dashed]
\tikzstyle{set}=[draw=primary,thick,ellipse,shade,top color=primary!50,bottom color=white,inner sep=4mm]
\tikzstyle{chain}=[%
    decoration={%
        shape backgrounds,
        shape=ellipse,
        shape width=5pt,
        shape height=2.5pt,
        shape sep=4pt},
    decorate
]
%    \end{macrocode}
%
%\begin{center}
%    \pgfsetlayers{background,main,foreground}
%    \begin{tikzpicture}%[baseline=(current bounding box.center)]
%        \node[point,label=180:$1$] (P1) {};
%        \node[point,label=180:$2$] (P2) [below=of P1] {};
%        \node[point,label=180:$3$] (P3) [below=of P2] {}; 
%        \draw (P1) 
%        ++ (3,0)  node[point,label=0:$4$] (P4) {}; 
%        \node[point,label=0:$5$] (P5) [below=of P4] {}; 
%        \node[point,label=0:$6$] (P6) [below=of P5] {};
%        \draw (P1) edge[->,bend left=20] (P4);
%        \draw (P2) edge[->,bend left=20] (P5);
%        \draw (P3) edge[->,bend left=20] (P6);
%        \begin{pgfonlayer}{background}
%            \node[set,fit=(P1) (P2) (P3)] {};
%            \node[set,fit=(P4) (P5) (P6)] {};
%        \end{pgfonlayer}
%    \end{tikzpicture}
%\end{center}
%
% \section{Color Schemes}
%
% \cn{definecolor} and \cn{colorlet} are provided by the \pkg{xcolor} package.
%
% This color scheme is ``\href{http://kuler.adobe.com/#themeID/123}{Raygun Gothic}'', which I made on Adobe's Kuler application.
%    \begin{macrocode}
\definecolor{primary}  {HTML}{3CFFEE}
\definecolor{secondary}{HTML}{24AABC}
\definecolor{tertiary}{HTML}{356781}
\definecolor{quartenary}{HTML}{2C3D51}
\definecolor{quintenary}{HTML}{1C1F24}
%    \end{macrocode}
%
% Here is a blue/green motif:
%    \begin{macrocode}
%:TODO make this a key
%\colorlet{primary}{blue!40!black}       % HTML {000066}
%\colorlet{secondary}{blue!20!white}     % HTML {CCCCFF}
%\colorlet{tertiary}{Green!40!Black}     % HTML {006600}
%\colorlet{quartenary}{Green!20!white}   % HTML {CCFFCC}
%\definecolor{quintenary}{HTML}{1C1F24}
\tikzstyle{curve}=[primary,thick]
\tikzstyle{curve label}=[black,opacity=1]
\tikzstyle{curve direction}=[curve,fill opacity=1]
\tikzstyle{region}=[curve,fill=quartenary]
\tikzstyle{surface}=[fill=quartenary,opacity=0.5]
\tikzstyle{function}=[curve]
%    \end{macrocode}
% \begin{center}
%   \begin{tikzpicture}
%      \tikzstyle{every node}=[minimum width=2cm,minimum height=1cm]
%      \node[fill=primary,label=above:primary] (primary) {};
%      \node[fill=secondary,right=0pt of primary,label=above:secondary] (secondary) {};
%      \node[fill=tertiary,right=0pt of secondary,label=above:tertiary] (tertiary) {};
%      \node[fill=quartenary,right=0pt of tertiary,label=above:quartenary] (quartenary) {};
%   \end{tikzpicture}
% \end{center}
%
% 
%
% \section{PGFPLOTS}
%
% \pgfplots/ is a massive \pgf//\tikz/ application that does graphs.  Very cool.
%    \begin{macrocode}
\RequirePackage{pgfplots}
\pgfplotsset{compat=1.10}
\usepgfplotslibrary{groupplots}
\usepgfplotslibrary{fillbetween}
%    \end{macrocode}
%
% We set every plot to draw axes through the origin 
% (The default is a box around the plot with ticks on the edges.)
% Also the ``cycle list'' of colors and line styles.
%    \begin{macrocode}
\pgfplotsset{
    every linear axis/.append style={%
        axis x line=middle, axis y line=middle,%
        cycle multi list={%
            solid,dashed,dotted\nextlist
            {function,primary},
            {function,secondary},
            {function,tertiary},
            {function,quartenary}
        },
    },
    every axis plot/.append style={every label/.append style={black}}
}
%    \end{macrocode}
%
% This is for marking points on plots.  It's a \tikz/ key used by \pgfplots/.
%    \begin{macrocode}
\tikzstyle{every mark}=[mark=*,mark size=1pt]
%    \end{macrocode}
%
%
% \subsection{Number lines}
%
% We can use pgfplots to make number lines, too.
% The \tk{number line} key does that.
% Thanks to TEX.SE user egreg for helping me out here. \url{http://tex.stackexchange.com/q/184191/1402}
%    \begin{macrocode}
\pgfplotsset{
    number line/.style={%
        axis x line=bottom,
        ymin=0,ymax=0.1,
        axis equal image,
        hide y axis,
        every axis x label/.append style={%
            anchor=west,
            align=left,
        },
        execute at end axis={
            \addplot[draw=none,forget plot] {0};
        }
    }
}
%    \end{macrocode}
% \begin{codeexample}[leave comments]
% \begin{tikzpicture}
%    \begin{axis}[
%       number line,
%       xmin=0,xmax=3,
%   ]
%    \end{axis}
% \end{tikzpicture}
% \end{codeexample}
% \changes{v0.3.1}{2015/07/16}{Added a \tk{y number line} key}
%    \begin{macrocode}
\pgfplotsset{%
    number line y/.style={%
        axis y line=left,
        axis equal image,
        xmin=0,xmax=0.1,
        hide x axis,
        every axis y label/.append style={
            anchor=south east,
            align=left
        },
%        execute at end axis={
%            \addplot[draw=none,forget plot] {0};
%        }
    }
}
%    \end{macrocode}
% \begin{codeexample}[leave comments]
% \begin{tikzpicture}
%    \begin{axis}[
%       number line y,
%       ymin=0,ymax=3,
%   ]
%    \end{axis}
% \end{tikzpicture}
% \end{codeexample}
% Use \tk{y number line} with \verb|scale=-1| if you want the number line to increase downwards.
% \begin{codeexample}[leave comments]
% \begin{tikzpicture}
%    \begin{axis}[
%       number line y,
%       ymin=0,ymax=3,
%       yscale=-1
%   ]
%    \end{axis}
% \end{tikzpicture}
% \end{codeexample}%
% \tk{interval labels} are for increasing/decreasing, concave up/down, etc.
%    \begin{macrocode}
\pgfplotsset{%
    interval labels/.style={
        after end axis/.prefix code={
            \foreach \x/\val/\desc in {#1} {
                \ifx\val\desc
                    \edef\temp{\noexpand\draw ({axis cs:\x,0}|-{xticklabel* cs:0}) 
                        node[number line value] {\unexpanded\expandafter{\val}};}%
                \else
                    \edef\temp{\noexpand\draw ({axis cs:\x,0}|-{xticklabel* cs:0}) 
                    node[number line value] {\unexpanded\expandafter{\val}}
                    node[number line description] {\unexpanded\expandafter{\desc}};}
                \fi
                \temp
            }
        }
    },
%    \end{macrocode}
% Point labels are for max/min, inflection point, etc.
%    \begin{macrocode}
    point labels/.style={
        after end axis/.prefix code={
            \foreach \x/\val/\desc in {#1} {
                \ifx\val\desc
                    \edef\temp{\noexpand\draw ({axis cs:\x,0} |- {xticklabel* cs:0}) 
                        node[number line value] {\unexpanded\expandafter{\val}};}%
                \else
                    \edef\temp{%
                        \noexpand\draw ({axis cs:\x,0} |- {xticklabel* cs:0}) 
                            node[number line value] {\unexpanded\expandafter{\val}};
                        \noexpand\draw ({axis cs:\x,0} |- {xticklabel cs:0})
                        node[number line description] {\unexpanded\expandafter{\desc\strut}};
                        }%
                \fi
                \temp
            }
        }
    }
}
%    \end{macrocode}
%
%
% \subsection{Polar plots}
%
% \begin{codeexample}[leave comments]
%    \begin{tikzpicture}
%    \tikzstyle{every plot}=[] % usually smooth, but that gets in the way of only marks below
%        \begin{polaraxis}[
%            axis x line=right,
%            xtick={0,30,45,60,90,120,135,150,180,210,225,240,270,300,315,330},
%            xticklabels={
%                0,$\frac{\pi}{6}$,$\frac{\pi}{4}$,$\frac{\pi}{3}$,$\frac{\pi}{2}$,$\frac{2\pi}{3}$,$\frac{3\pi}{4}$,$\frac{5\pi}{6}$,
%                $\pi$,$\frac{7\pi}{6}$,$\frac{5\pi}{4}$,$\frac{4\pi}{3}$,$\frac{3\pi}{2}$,$\frac{5\pi}{3}$,$\frac{7\pi}{4}$,$\frac{11\pi}{6}$
%            },
%            ymin=0,ymax=2,
%        ]
%            \addplot+[mark=none,data cs=polarrad,domain=0:6.28,forget plot,samples=100] {2*sin(deg(x))};
%            \addplot+[only marks,data cs=polarrad,mark=*] coordinates{
%                (0.5236,1.)
%                (0.7854,1.4142)
%                (1.0472,1.7321)
%                (1.5708,2.)
%                (2.0944,1.7321)
%                (2.3562,1.4142)
%                (2.618,1.)
%                (3.1416,0.)
%            };
%        \end{polaraxis}
%    \end{tikzpicture}
% \end{codeexample}
%    \begin{macrocode}
\usepgfplotslibrary{polar}
%    \end{macrocode}
%
% \subsection{Slope fields}
% 
% Slope fields are implemented as \pgfplots/ ``quivers''---that is, vector fields.
% But since no arrow style is set the arrows are headless.
% We use a coordinate filter to shift the line segments so that the base point 
% is at the \emph{center} of the segment instead of the tail.
% \begin{codeexample}[leave comments]
%    \begin{tikzpicture}[
%        declare function={
%            F(\y,\t) = \t+\y;
%            phi(\t,\yo) = (\yo+1)*exp(\t) - \t - 1;
%        }
%    ]
%        \begin{axis}[
%            xmin=-3,xmax=3,
%            ymin=-3,ymax=3,
%            axis equal image,
%            view={0}{90},
%            xtick={-3,-2,...,3},
%            ytick={-3,-2,...,3},
%        ]
%            \addplot3+[
%                slope field={F(x,y)},
%                domain=-3:3,
%                domain y=-3:3,
%            ]{x};
%            \addplot+[ode solution,variable=\t,domain=-3:3] {phi(t,1)};
%            \addplot+[ode solution,variable=\t,domain=-3:3] {phi(t,0)};
%        \end{axis}
%    \end{tikzpicture}
% \end{codeexample}
%
%    \begin{macrocode}
\pgfplotsset{
slope field/.style={
    secondary,thin,
    /pgfplots/quiver={u=1,v={#1},scale arrows=0.1},
    /pgfplots/x filter/.expression={x-1*0.5*\pgfkeysvalueof{/pgfplots/quiver/scale arrows}},
    /pgfplots/y filter/.expression={y-(#1)*.5*\pgfkeysvalueof{/pgfplots/quiver/scale arrows}}
    },
    ode solution/.style={primary},
}
%    \end{macrocode}
%
% \section{Vectors}
%
% \begin{TikZkey}{vector}
%     The \tk{vector} key is for vectors.
% \changes{v0.4}{2015/08/27}{added a \tk{vector} key}
% \begin{codeexample}[leave comments]
%    \begin{tikzpicture}
%        \draw[vector] (0,0) --++ (3,2) node[vector label,below] {$\mathbf{u}$};
%    \end{tikzpicture}
% \end{codeexample}
%    \begin{macrocode}
\tikzset{>=stealth}
\tikzstyle{vector}=[primary,very thick,arrows={->}]
%    \end{macrocode}
% \end{TikZkey}
% \begin{TikZkey}{vector label}
% \changes{v0.5}{2015/09/18}{Added}
%    \begin{macrocode}
\tikzstyle{vector label}=[pos=0.5,black]
%    \end{macrocode}
% \end{TikZkey}
%
%
%
% This is the end of \fn{tikzlibrarynyucalc.code.tex}.
%    \begin{macrocode}
%</library>
%    \end{macrocode}
%\Finale
